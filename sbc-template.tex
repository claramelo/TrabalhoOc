\documentclass[12pt]{article}

\usepackage{sbc-template}
\usepackage{graphicx,url}

\usepackage[brazil]{babel} 

\usepackage[utf8]{inputenc}   
\usepackage{color}


     
\sloppy

\title{Trabalho de OC}

\author{Ana Vitória V. Cordeiro, Felipe Manzoni, Maria Clara}


\address{Instituto de Computação -- Universidade Federal do Amazonas
  (UFAM)\\
  Manaus -- AM -- Brasil\\avvc@icomp.ufam.edu.br
}

\begin{document} 

\maketitle

\maketitle
\begin{abstract}
	This paper presents ARM's architecture by describing some its issues,  specially the set of instructions.  ARM programming issues and the structure of its instructions are described in details. Moreover, the following subjects are also approched: ARM's memory hierarchy, ARM use cases and the ARM insertion in the market of processors. 
\end{abstract}

\begin{resumo} 
	Este artigo aborda os principais aspectos da arquitetura ARM, descrevendo as características dessa arquitetura e dando ênfase ao seu conjunto de instruções, tanto na parte de programação quanto no que se refere a estrutura de cada instrução. O artigo, também, aborda temas como: hierarquia de memória da arquitetura ARM, a aplicação do ARM, a sua inserção no mercado de processadores e uma comparação com o seu concorrente, o MIPS.
\end{resumo}

\section{Introdução}

A arquitetura ARM é um conjunto de instruções arquitetônicas para processadores, desenvolvida pela companhia britânica \textit{ARM Holdings} em 1980. Os primeiros produtos baseados em ARM foram coprocessadores desenvolvidos para a série de  microcomputadores da BBC. Ao longo do tempo, a arquitetura ARM evoluiu para incluir características arquitetônicas que atendessem à crescente demanda por novas funcionalidades, alta performance e as necessidades de novos mercados emergentes.

Estruturalmente a arquitetura ARM é baseada na arquitetura RISC, logo, utiliza um conjunto pequeno e altamente otimizado de instruções, em vez de um conjunto mais especializado encontrado em outros tipos de arquiteturas. Ter um conjunto de instruções neste formato faz com que o processador que utiliza essa arquitetura seja mais veloz em relação a processadores que utilizam a arquitetura CISC, além de ter um número de ciclos de clock menor, pois leva menos tempo para selecionar uma instrução.

Este artigo aborda conceitos importantes da arquitetura ARM e traz também uma explanação sobre as aplicações dos processadores ARM e a sua inserção no mercado tecnológico, além de compará-lo ao processador MIPS. Para que haja um melhor entendimento do assunto, o artigo foi dividido em cinco seções, são elas: história, arquitetura do conjunto de instruções, \textcolor{green}{hierarquia de memória}, aplicações e comparação entre MIPS e ARM.

Antes de tratarmos da estrutura e do funcionamento da arquitetura ARM é importante compreendermos os princípios que foram usados no projeto de desenvolvimento da arquitetura. Com o decorrer dos anos, a arquitetura ARM passou por diversas evoluções ocasionando mudanças não somente na sua estrutura, mas também nos processadores ARM, ocasionando o surgimento das famílias de processadores. Todos estes aspectos fazem parte da história da arquitetura ARM, por isso serão abordados com maiores detalhes na primeira seção deste artigo.

A arquitetura do conjunto de informações do ARM será tratada na segunda seção do artigo. Nesta, apresentaremos as instruções mais importantes da arquitetura ARM, descrevendo as suas características e o seu formato. Os princípios básicos adotados na elaboração do conjunto de instruções serão abordados, pois é de extrema importância que seja conhecida a base para a construção das instruções.

Com o desenvolvimento da tecnologia os processadores ARM conquistaram cada vez mais espaço no mercado. A necessidade de que os produtos eletrônicos apresentassem alta performance fez com que os fabricantes optassem por um processador que garantisse um bom desempenho para os aparelhos e um baixo consumo de energia. Como estas são as principais características dos processadores ARM eles foram escolhidos para suprir o mercado de eletroeletrônicos, o que elevou a qualidade dos produtos e tornou-os mais competitivos. Os processadores ARM são utilizados em smatphones, PDAs, câmeras digitais, DVDs Blu-Ray, Wireless Lan, Bluetooth, automóveis entre vários outros dispositivos. As aplicações dos processadores ARM serão abordadas com mais detalhes na quarta seção deste artigo.

A evolução da tecnologia ocasionou o surgimento de uma nova arquitetura, a arquitetura MIPS. MIPS é um acrônimo para microprocessor without interlocked pipeline stages, ou seja, microprocessador sem estágios interligados de pipeline. Na quinta seção do artigo será feita uma comparação entre o ARM e o MIPS, expondo as diferenças entre essas arquiteturas e seu conjunto de instruções, montrando os pontos positivos e negativos de se utilizar processadores projetados com essas arquiteturas.   



\bibliographystyle{sbc}
\bibliography{sbc-template}

\end{document}
